% Corrected Abstract for CodeSense AI
% Copy and paste this into Overleaf.com

\begin{abstract}
Code analysis and quality assessment are essential components of modern software development, ensuring that code is not only functionally correct but also maintainable, secure, and aligned with industry best practices. Traditional approaches to code evaluation often rely on fragmented toolchains, manual reviews, or basic static analysis that cannot provide comprehensive, real-time feedback combining execution validation with intelligent analysis. In this work, we present CodeSense AI, an integrated web-based platform that combines secure containerized code execution, AI-powered intelligent analysis, and quantitative complexity assessment to provide comprehensive code quality evaluation in real-time.

The system leverages Docker containerization technology to execute user-submitted code securely and in complete isolation across multiple programming languages, including Python, JavaScript, Java, C++, and Go. Each execution environment is constrained with appropriate resource limits and security measures to prevent system compromise while enabling safe analysis of untrusted code. Upon successful code execution, the platform integrates Google Gemini AI (gemini-flash-latest model) to provide context-aware, intelligent analysis that goes beyond traditional static analysis. The AI component analyzes both the source code and execution results to generate actionable feedback, identifying potential issues, suggesting optimizations, and providing educational insights about code quality and best practices.

To complement AI-driven analysis, CodeSense AI incorporates the Lizard library for automated cyclomatic complexity calculation, providing quantitative metrics that help developers understand code maintainability, structural complexity, and potential refactoring opportunities. The platform combines these qualitative AI insights with quantitative complexity metrics to deliver comprehensive code quality assessment that addresses both immediate functionality and long-term maintainability concerns.

The system features a modern web-based interface built with FastAPI, providing RESTful APIs for code submission, analysis retrieval, and user management. Secure user authentication using JWT tokens enables personalized experiences, submission history tracking, and progress monitoring. All user data, code submissions, and analysis results are persistently stored in a relational database using SQLAlchemy ORM, ensuring data integrity and enabling historical analysis of coding patterns and improvement trends.

CodeSense AI addresses critical gaps in existing code analysis tools by providing immediate, comprehensive feedback that combines execution validation, AI-powered insights, and quantitative metrics in a single, accessible platform. The system is particularly valuable for educational environments where students require immediate feedback to accelerate learning, as well as for individual developers and small teams seeking comprehensive code quality assessment without the overhead of complex toolchain management.

Empirical evaluation demonstrates that CodeSense AI successfully provides secure, multi-language code execution with comprehensive analysis capabilities, achieving response times suitable for interactive use while maintaining strict security isolation. The platform's hybrid approach—combining containerized execution, AI analysis, and complexity metrics—provides more comprehensive feedback than traditional tools that focus on single aspects of code quality. The web-based architecture ensures accessibility across different platforms and devices, while the modular design supports future extensions and additional programming language support.

Overall, CodeSense AI represents a significant advancement in accessible code analysis tools by unifying secure execution, intelligent AI analysis, and quantitative assessment in a single platform. It empowers developers and students to improve code quality systematically while providing immediate, actionable feedback that supports both learning and professional development. The platform establishes a foundation for future innovations in AI-assisted software development and educational technology.
\end{abstract}
